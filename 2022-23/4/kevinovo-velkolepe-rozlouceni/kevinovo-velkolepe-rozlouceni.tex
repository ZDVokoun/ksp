\documentclass{fkssolpub}

\usepackage[czech]{babel}
\usepackage{fontspec}
\usepackage{fkssugar}

\author{Ondřej Sedláček}
\school{Gymnázium Oty Pavla} 
\series{35-4}
\problem{2} 

\begin{document}

Můj algoritmus je rozdělen na dva různé části. V první části prohazíme tašky
tak, abychom celý řetězec otočily (reverse). To probíhá tak, že vybereme vždy první a poslední
neprohozenou tašku v řadě a takhle prohazujeme, dokud se nedostaneme do poloviny
řetězce. Pro tuto část si musí Kevin musí zapamatovat dvě různá čísla a složitost
bude lineární.

V druhé části budeme následně otáčet jednotlivá slova. To uděláme tak, že v každém
kroku najdeme první mezeru v nezpracované části řetězce, nebo dojdeme na konec, 
čímž ohraničíme první slovo v nezpracovaném řetězci, a to slovo pak otočíme stejně
jako v první části celý řetězec. Kevinovi pak postačí se vždycky zapamatovat
jen čtyři čísla -- začátek nezpracovaného řetězce, konec prvního slova v řetězci
a pak dvě čísla pro ototáčení slova. A protože složitost otáčení a hledání mezer je 
lineární, složitost druhé části je celá lineární.

Celý tento algoritmus má prostorovou složitost $\mathcal{O}(1)$ a časovou
složitost $\mathcal{O}(n)$. Zároveň všechna ukládaná čísla nejsou větší než délka
řetězce, díky čemuž tento algoritmus splňuje podmínku ze zadání.

  
\end{document}
