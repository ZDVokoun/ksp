\documentclass{fkssolpub}

\usepackage[czech]{babel}
\usepackage{fontspec}
\usepackage{fkssugar}
\usepackage{amsmath}

\author{Ondřej Sedláček}
\school{Gymnázium Oty Pavla} 
\series{35-4}
\problem{S} 

\begin{document}

\section{Úkol 1 -- Mnohoúhelník}

Díky konvexitě mnohoúhelníku nám k rozhodnutí, zda bod $\mathbf{x}$ leží uvnitř
mnohoúhelníku, stačí pro všechny hrany zkontrolovat, jestli bod $\mathbf{x}$ leží
na stejné straně od každé hrany. To můžeme udělat tak, že získáme vektor hrany
mezi vrcholy $\mathbf{p}_i$ a $\mathbf{p}_{i+1}$ jako 
$\mathbf{h}_{i,i+1} = \mathbf{p}_{i+1} - \mathbf{p}_i$ a nový bod $\mathbf{x}_{i,i+1} = 
\mathbf{x} - \mathbf{p}_i$. Teď získáme nový vektor $\mathbf{u}_{i,i+1} = 
\big( \begin{smallmatrix}
  0 & -1 \\ 1 & 0 
\end{smallmatrix} \big) \cdot \mathbf{h}_{i,i+1}$, a vůči němu vypočítáme skalární
součin $a_{i,i+1} = \langle \mathbf{u}_{i,i+1}, \, \mathbf{x}_{i,i+1} \rangle$. Pak 
znaménko skalárního součinu $a_{i,i+1}$ nám určí, na jaké straně hrany bod 
$\mathbf{x}$ leží.

Tento algoritmus pak pro každou sousední dvojici vrcholů $\mathbf{p}_i$ a 
$\mathbf{p}_{i+1}$ (a dvojici $\mathbf{p}_n$ a $\mathbf{p}_0$) spočítá 
skalární součin $a_{i, i+1}$ a pokud je ten skalární součin nenulový, zkontroluje,
jestli všechny ostatní mají stejné znaménko. Kontrolu bude provádět tak, že
se uloží znaménko skalárního součtu první dvojice, jehož skalární součet je
nenulový, a pak pro každou další dvojici s nenulovým skalárním součinem se 
znaménko porovnává s uloženým znaménkem. Protože čas pro kontrolu jedné hrany
je konstantní, je časová složitost tohoto algoritmu $\mathcal{O}(n)$.

\section{Úkol 2 -- Vzdálenost od roviny}

Protože potřebujeme, aby rovina procházela počátkem, získáme novou rovinu $\rho'$
a nový bod $\mathbf{x}'$, pro které platí:

\[
  \rho' = \rho - \mathbf{c} = a_1 \mathbf{u}_1 + a_2 \mathbf{u}_2
\]
\[
  \mathbf{x}' = \mathbf{x} - \mathbf{c} = (5,1,0)^T
\]

Teď musíme získat ortogonální bázi roviny $\rho'$, kterou označím $\mathbf{v}_1$ a
$\mathbf{v}_2$. Bazický vektor $\mathbf{v}_1$ dostaneme jednoduše normalizací
vektoru $\mathbf{u}_1$:

\[
  \mathbf{v}_1 = \frac{\mathbf{u}_1}{\left\Vert \mathbf{u}_1 \right\Vert}
\]

Bazický vektor $\mathbf{v}_2$ pak dostaneme, jak je popsán v článku:

\[
  \mathbf{n}_2 = \mathbf{u}_2 - \langle \mathbf{u}_1, \mathbf{v}_1 \rangle
    \mathbf{v}_1
\]
\[
  \mathbf{v}_2 = \frac{\mathbf{n}_2}{\left\Vert \mathbf{n}_2 \right\Vert}
\]

Teď nám zbývá najít projekci:

\[
  \mathbf{x}'_{\rho} = \langle \mathbf{x}', \mathbf{v}_1 \rangle \mathbf{v}_1 + 
    \langle \mathbf{x}', \mathbf{v}_2 \rangle \mathbf{v}_2
\]

Po dosazení všech hodnot určíme vzdálenost
$\left\Vert \mathbf{x}' - \mathbf{x}'_{\rho} \right\Vert = 3,15$.

\section{Úkol 3 -- Fyzikální měření}

Nejdříve z naměřených hodnot potřebujeme získat lineární systém rovnic.
Všechny rovnice budou ve tvaru $pt + q = m$, kde potřebujeme zjistit hodnoty
$p$ a $q$, abychom získali lineární funkci.

Proto matice $\mathbf{A}$, vektor $\mathbf{b}$ a vektor $\mathbf{x}$ budou následující:

\[
  \mathbf{A} = \begin{pmatrix}
    0 & 3 & 10 & 14 & 20 \\
    1 & 1 & 1 & 1 & 1
  \end{pmatrix}^T
  \qquad
  \mathbf{b} = (67, 66, 63, 62, 60)^T \qquad \mathbf{x} = \begin{pmatrix}
    p \\ q
  \end{pmatrix}
\]

Teď musíme akorát dosadit hodnoty do vzorce $\mathbf{x} = 
(\mathbf{A}^T \mathbf{A})^{-1} \mathbf{A}^T \mathbf{b}$. Pak nám vyjdou
hodnoty:

\[
  p \doteq -0,35 \qquad q \doteq 66,93
\]

Číslo $p$ nám určuje přibližnou změnu hmotnosti za čas, proto rychlost vypařování
bude $0,35 \, \text{g} \cdot \text{s}^{-1}$.
  
\end{document}
