\documentclass{fkssolpub}

\usepackage[czech]{babel}
\usepackage{fontspec}
\usepackage{fkssugar}

\author{Ondřej Sedláček}
\school{Gymnázium Oty Pavla} 
\series{35-4}
\problem{4} 

\begin{document}

Nejdříve popíšu algoritmus na vyřešení tohoto problému pro orientovaný acyklický
graf (DAG) a později pak zobecním na všechny orientované grafy.

Pro DAG je zásadní následující pozorování. Když se podíváme na DAG, který je
topologicky seřazen, pak můžeme všimnout, že aby byl celý graf silně souvislý, 
potřebujeme najít všechny vrcholy bez výstupních hran a všechny vrcholy 
bez vstupních hran a ty propojit hranami tak, aby každý měl alespoň
jednu vstupní a alespoň jednu výstupní hranu. Proto pro vyřešení tohoto problému 
musíme spočítat počet vrcholů bez vstupních a počet vrcholů bez výstupních hran 
a vrátit jejich maximum. To můžeme udělat tak, že navštívíme všechny
vrcholy, postupně pro každý vrchol určíme, jestli má alespoň jednu vstupní hranu
a jestli má alespoň jednu výstupní hranu, a pak z pole zjistíme počet vrcholů bez 
vstupních či výstupních hran. To má časovou složitost $\mathcal{O}(V + E)$ a
prostorovou složitost $\mathcal{O}(V)$.

Pro obecný orientovaný graf musíme nejprve převést vstupní graf právě na
DAG, díky čemuž budeme použít algoritmus výše. Na to musíme najít
všechny komponenty silné souvislosti a každou z nich nahradit vrcholem.
Na to můžeme použít Tarjanův algoritmus, jehož časová složitost je
$\mathcal{O}(V + E)$ a jehož prostorová složitost je $\mathcal{O}(V + E)$. 
Nahrazování komponent vrcholy trvá stejně dlouho.

Proto celý tento algoritmus běží v čase $\mathcal{O}(V + E)$ a její prostorová složitost
je $\mathcal{O}(V + E)$.
  
\end{document}
