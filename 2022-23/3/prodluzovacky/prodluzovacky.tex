\documentclass{fkssolpub}

\usepackage[czech]{babel}
\usepackage{fontspec}
\usepackage{fkssugar}

% Čmajznuto z https://tex.stackexchange.com/questions/536081/unicode-characters-misplaced-within-lstlistings-environment
% \makeatletter
% \lst@InputCatcodes
% \def\lst@DefEC{%
%  \lst@CCECUse \lst@ProcessLetter
%   ^^80^^81^^82^^83^^84^^85^^86^^87^^88^^89^^8a^^8b^^8c^^8d^^8e^^8f%
%   ^^90^^91^^92^^93^^94^^95^^96^^97^^98^^99^^9a^^9b^^9c^^9d^^9e^^9f%
%   ^^a0^^a1^^a2^^a3^^a4^^a5^^a6^^a7^^a8^^a9^^aa^^ab^^ac^^ad^^ae^^af%
%   ^^b0^^b1^^b2^^b3^^b4^^b5^^b6^^b7^^b8^^b9^^ba^^bb^^bc^^bd^^be^^bf%
%   ^^c0^^c1^^c2^^c3^^c4^^c5^^c6^^c7^^c8^^c9^^ca^^cb^^cc^^cd^^ce^^cf%
%   ^^d0^^d1^^d2^^d3^^d4^^d5^^d6^^d7^^d8^^d9^^da^^db^^dc^^dd^^de^^df%
%   ^^e0^^e1^^e2^^e3^^e4^^e5^^e6^^e7^^e8^^e9^^ea^^eb^^ec^^ed^^ee^^ef%
%   ^^f0^^f1^^f2^^f3^^f4^^f5^^f6^^f7^^f8^^f9^^fa^^fb^^fc^^fd^^fe^^ff%
%   ^^^^03b6% <--- for ζ
%   ^^00}
% \lst@RestoreCatcodes
% \makeatother

\usepackage{listings}
\lstset{
  basicstyle=\ttfamily,
  mathescape,
  extendedchars=true,
}

\author{Ondřej Sedláček}
\school{Gymnázium Oty Pavla} 
\series{35-3}
\problem{3} 

\begin{document} 

V této úloze se nabízí díky tomu, že pracujeme se stromem, napsat algoritmus 
rekurzivně. 

Nejprve budeme uvažovat, jak pro jednu prodlužovačku zjistíme, jaké zařízení
vypojit, když do této prodlužovačky jsou připojeny jen zařízení. Na to
použijeme následující postup: nejprve zjistíme příkony zapojených zařízení
a jejich součet, a pak odpojujeme zařízení s maximálním příkonem,
dokud není součet menší než maximální příkon, protože tím jedině
dosáhneme nejnižšímu počtu kroků.

Tím se dostáváme k případu, kdy do prodlužovačky bude zapojena další
prodlužovačka. Protože když budeme znát všechna zařízení zapojené přímo
či nepřímo do zapojené prodlužovačky, můžeme tento případ vyřešit stejně
jako kdyby všechna zařízení byla zapojena rovnou do zpracovávané
prodlužovačky. Tohle však můžeme udělat, jakmile všechny dceřiné prodlužovačky
jsou zpracované, proto budeme zpracovávat od spoda nahoru, aby byl výsledek
správný.

Implementaci tohoto algoritmu můžete vidět v pseudokódu níže. Pro
efektivní získávání maxima využívám maximovou haldu. K nejhorší
časové složitosti pak dojde, když se všechna zařízení odpojí
v prvních prodlužkách, protože v každé vrstvě dojde ke slévání
hald, což celkem potrvá $\mathcal{O}(N L)$ pro počet vrstev $L$ a pro počet
zařízení $N$, a odpojení všech zařízení $\mathcal{O}(N \log N)$.

\begin{lstlisting}

Funkce prodluzovacky(vrchol v, int &pocetOdpojenych, nafPole<int> &kOdpojeni) 
    -> (maxHalda<(int, int)>, int):
  // zarizeni budu zpracovavat jako pary dvou cisel, kde prvni
  // je pozadovany prikon a druhy ID zarizeni, abych mohl vypsat,
  // jake presne zarizeni vypojit
  maxHalda<(int, int)> h 
  int s = 0 // suma
  (int,int) max // aktualni zarizeni s maximalnim prikonem
  Pokud v.deti.pocet = 0: // kdyz se jedna o zarizeni (list)
    h.pridat((v.prikon, v.cislo))
    s <- s + v.prikon
    Vratit (h, s)
  Jinak:
    Pro u ve v.deti:
      (maxHalda<(int, int)>, int) vysledek = prodluzovacky(
        u, pocetOdpojenych, kOdpojeni
      )
      h.spojitPole(vysledek[0]) // predpokladam, ze halda je ukladana v poli
      s <- s + vysledek[1]
    h.vyrovnat();
  Dokud v.P < s:
    max <- h.extrahovatMax()
    s <- s - max[0]
    kOdpojeni.pridat(max[1])
  Vratit (h, s)

int pocetOdpojenych
nafPole<int> kOdpojeni
prodluzovacky(koren, pocetOdpojenych, kOdpojeni)
Vratit (pocetOdpojenych, kOdpojeni)

\end{lstlisting}

\end{document}
