\documentclass{fkssolpub}

\usepackage[czech]{babel}
\usepackage{fontspec}
\usepackage{fkssugar}
\usepackage{amsmath}

\author{Ondřej Sedláček}
\school{Gymnázium Oty Pavla} 
\series{35-2}
\problem{S} 

\begin{document}

\section{Úkol 1 -- Nelineární zobrazení}

Jeden z jednoduše popsatelných případů je přičítání konstantního
vektoru, aneb "posouvání". U něj neplatí ani jedna z podmínek.

Mějme zobrazení $f(\mathbf{x}) = \mathbf{x} + \mathbf{c}$ s parametrem
$\mathbf{c} \in \mathbb{R}^2$. Pro každou z podmínek vyzkoušíme, jestli platí
vždy:

\[
  f(\mathbf{x} + \mathbf{y}) = f(\mathbf{x}) + f(\mathbf{y})
\]
\[
  \mathbf{x} + \mathbf{y} + \mathbf{c} = \mathbf{x} + \mathbf{c} + \mathbf{y} + \mathbf{c}
\]
\[
  \mathbf{0} = \mathbf{c}
\]

Tato podmínka tedy nemůže platit vždy. S druhou podmínkou to bude podobné:

\[
  f(a \mathbf{x}) = a f(\mathbf{x})
\]
\[
  a \mathbf{x} + \mathbf{c} = a \mathbf{x} + a \mathbf{c}
\]
\[
  \mathbf{c} = a \mathbf{c}
\]

Jeden z příkladů, na kterém je neplatnost vidět, je třeba nulový vektor.


\section{Úkol 2 -- Fibonacciho čísla}

Matice $\mathbf{Q}$ lze jednoduše odvodit z podmínky. Protože v horním
řádku chceme $F_n$ nahradit $F_{n+1}$ a ve spodním chceme součet
předchozích dvou čísel posloupnosti, výsledná matice $\mathbf{Q}$ musí
být:

\[
  \mathbf{Q} = \begin{pmatrix}
    0 & 1 \\ 1 & 1
  \end{pmatrix}
\]

Můj algoritmus se nachází v souboru \verb|fibonacci.jl| 
a je napsaný v jazyce Julia. Díky tomu, že
postupně počítám matice pro mocniny dvojky a ty
podle binárního zápisu skládám, dostáváme časovou složitost
$\mathcal{O}(\log n)$ a prostorová složitost je $\mathcal{O}(1)$.

\section{Úkol 3 -- Dosažitelnost}

Předchozí postup využívající maticový součin stačí jen mírně upravit,
a to interpretováním pomocí Booleovské algebry, neboli jen 
nahrazením součtu operátorem OR a násobení operátorem AND.
Pak dostaneme matici dosažitelnosti. Tento postup bude asymptoticky
trvat stejně dlouho jako získání matice počtu sledů.

\section{Úkol 4 -- Obecná inverze}

Inverzní matici $\mathbf{A}^{-1}$ dostaneme po vyřešení dvou lineárních
soustav rovnic pro každý sloupec jednotkové matice.

Pro první sloupec máme soustavu:

\[
  \begin{pmatrix}
    c & -c \\ c & c
  \end{pmatrix} \mathbf{x}_1 = \begin{pmatrix}
    1 \\ 0
  \end{pmatrix}
\]

Jejíž řešení je:

\[
  \begin{pmatrix}
    2c & 0 \\ 0 & c
  \end{pmatrix} \mathbf{x}_1 = \begin{pmatrix}
    1 \\ -\frac{1}{2}
  \end{pmatrix}
\]
\[
  \mathbf{x}_1 = \frac{1}{2c} \begin{pmatrix}
    1 \\ -1
  \end{pmatrix}
\]

A pro druhý sloupec:

\[
  \begin{pmatrix}
    c & -c \\ c & c
  \end{pmatrix} \mathbf{x}_2 = \begin{pmatrix}
    0 \\ 1
  \end{pmatrix}
\]

Jejíž řešení je:

\[
  \begin{pmatrix}
    2c & 0 \\ 0 & c
  \end{pmatrix} \mathbf{x}_2 = \begin{pmatrix}
    1 \\ \frac{1}{2}
  \end{pmatrix}
\]
\[
  \mathbf{x}_2 = \frac{1}{2c} \begin{pmatrix}
    1 \\ 1
  \end{pmatrix}
\]

Proto inverzní matice $\mathbf{A}^{-1}$ je:

\[
  \mathbf{A}^{-1} = \frac{1}{2c} \begin{pmatrix}
    1 & 1 \\ -1 & 1
  \end{pmatrix}
\]


\end{document}
