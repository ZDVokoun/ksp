\documentclass{fkssolpub}

\usepackage[czech]{babel}
\usepackage{fontspec}
\usepackage{fkssugar}
\usepackage{amsmath}

\author{Ondřej Sedláček}
\school{Gymnázium Oty Pavla} 
\series{35-1}
\problem{S} 

\begin{document}

\section{Úkol 1 -- Robot na Marsu}

Abychom zkalibrovali vesmírnou sondu, musíme vyřešit tyto dvě rovnice:

\[
  \begin{pmatrix}
    1 \\ 0
  \end{pmatrix} = f_1 \mathbf{u} + f_2 \mathbf{v} 
\]
\[
  \begin{pmatrix}
    0 \\ 1
  \end{pmatrix} = l_1 \mathbf{u} + l_2 \mathbf{v} \text{, }
\]

kde $\mathbf{u}$ a $\mathbf{v}$ jsou nové vektory pohybu a koeficienty
$f_1$, $f_2$ pro pohyb dopředu a $l_1$, $l_2$ pro pohyb doleva. Každá z těchto
rovnic lze zapsat jako soustavu rovnic, ze který pak lze koeficienty vyjádřit.
Její řešení jsou (pro pohyb dopředu s podmínkou $v_2 \neq 0$ a pro pohyb doleva
$v_1 \neq 0$):

\[
  f_1 = \frac{v_2}{u_1 v_2 - u_2 v_1}
\]
\[
  f_2 = -\frac{u_2}{u_1 v_2 - u_2 v_1}
\]
\[
  l_1 = \frac{v_1}{u_2 v_1 - u_1 v_2}
\]
\[
  l_2 = - \frac{u_1}{u_2 v_1 - u_1 v_2}
\]

\section{Úkol 2 -- Nezávislost a soustavy}

Vzorec lineární kombinace lze pomocí roznásobení a sečtení složek převést
na soustavu rovnic, která se pak vyřeší za pomoci Gaussovy eliminace.

Pokud chceme zjistit, jestli jsou vektory $\mathbf{x}_1, ..., \mathbf{x}_n$
lineárně nezávislé, musíme najít jiné řešení rovnice:

\[
  \sum_{i=1}^n a_i \mathbf{x}_i = \mathbf{0},
\]

kterou převedeme na soustavu rovnic, než když jsou všechny koeficienty
$a_i = 0$. Pokud najdeme po vyřešení soustavy rovnic pomocí Gaussovy eliminace 
jiné řešení, pak jsou vektory lineárně závislé.

Pokud chceme některý z vektorů vyjádřit jako lineární kombinaci zbytku, pak si
musíme vybrat jeden z těchto vektorů a pak rovnici lineární kombinace upravit
vynásobením složek koeficienty a sečtení složek, čímž získáme znova soustavu
rovnic řešitelnou Gaussovou eliminací. Pokud jsme si však vybrali takový vektor,
který není v lineárním obalu zbytku, budeme ho muset vyměnit za jiný a postup
zopakovat.

\section{Úkol 3 -- Algoritmus eliminace}

Implementace Gaussovy eliminace se nachází v souboru \verb|gauss.py|.
Asymptotická časová složitost této implementace je $\mathcal{O}(n^3)$.

\end{document}
