\documentclass{fkssolpub}

\usepackage[czech]{babel}
\usepackage{fontspec}
\usepackage{fkssugar}
\usepackage{amsmath}

\author{Ondřej Sedláček}
\school{Gymnázium Oty Pavla} 
\series{35-1}
\problem{S} 

\begin{document}

\section{Úkol 1 -- Robot na Marsu}

Abychom zkalibrovali vesmírnou sondu, musíme vyřešit tyto dvě rovnice:

\[
  \begin{pmatrix}
    1 \\ 0
  \end{pmatrix} = f_1 \mathbf{u} + f_2 \mathbf{v} 
\]
\[
  \begin{pmatrix}
    0 \\ 1
  \end{pmatrix} = l_1 \mathbf{u} + l_2 \mathbf{v} \text{, }
\]

kde $\mathbf{u}$ a $\mathbf{v}$ jsou nové vektory pohybu a koeficienty
$f_1$, $f_2$ pro pohyb dopředu a $l_1$, $l_2$ pro pohyb doleva. Každá z těchto
rovnic lze zapsat jako soustavu rovnic, ze který pak lze koeficienty vyjádřit.
Její řešení jsou (pro pohyb dopředu s podmínkou $v_2 \neq 0$ a pro pohyb doleva
$v_1 \neq 0$):

\[
  f_1 = \frac{v_2}{u_1 v_2 - u_2 v_1}
\]
\[
  f_2 = -\frac{u_2}{u_1 v_2 - u_2 v_1}
\]
\[
  l_1 = \frac{v_1}{u_2 v_1 - u_1 v_2}
\]
\[
  l_2 = - \frac{u_1}{u_2 v_1 - u_1 v_2}
\]

\section{Úkol 2 -- Nezávislost a soustavy}



\section{Úkol 3 -- Algoritmus eliminace}

Implementace Gaussovy eliminace se nachází v souboru \verb|gauss.py|.
Asymptotická časová složitost této implementace je $\mathcal{O}(n^3)$.

\end{document}
