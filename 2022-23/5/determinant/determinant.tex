\documentclass{fkssolpub}

\usepackage[czech]{babel}
\usepackage{fontspec}
\usepackage{fkssugar}
\usepackage{amsmath}

\author{Ondřej Sedláček}
\school{Gymnázium Oty Pavla} 
\series{35-5}
\problem{S} 

\begin{document}

\section{Úkol 1 -- Obsah mnohoúhelníku}

Zde využijeme faktu, že za pomoci determinantu zjišťujeme orientovaný obsah.
Díky němu můžeme postupovat tak, že pro každou hranu zjistíme obsah trojúhelníku,
který bude mít za vrcholy počátek a vrcholy hrany. Když tyto obsahy sečteme,
orientovanost nám zajistí, že přebývající obsah se vyruší, čímž nám zůstane
jen obsah mnohoúhelníku (buď jako kladné či záporné číslo).

Můj algoritmus teda funguje tak, že pro každý dva sousední vrcholy spočítáme
determinant (($p_n$, $p_0$) a ($p_k$, $p_{k+1}$) pro $k \in \mathbb{N}$ 
a $k < n$) ve stejném směru, ty postupně sečteme a po sečtení vrátíme pro součet $x$
$\left|\frac{x}{2}\right|$. Složitost tohoto algoritmu bude pak zřejmě
$\mathcal{O}(n)$.

\section{Úkol 2 -- Polynom}

Nejdříve musíme zapsat soustavu rovnic, které řešíme, maticově:

\[
  \begin{pmatrix}
    x_0^2, x_0, 1 \\ x_1^2, x_1, 1 \\ x_2^2, x_2, 1
  \end{pmatrix} 
  \begin{pmatrix}
    a \\ b \\ c
  \end{pmatrix} = \begin{pmatrix}
    y_0 \\ y_1 \\ y_2
  \end{pmatrix}
\]

Z níž už vidíme, u jaké matice potřebujeme zjistit determinant. Nejdříve zjistíme
permutační tvar:

\[
  \text{det} \begin{pmatrix}
    x_0^2, x_0, 1 \\ x_1^2, x_1, 1 \\ x_2^2, x_2, 1
  \end{pmatrix} = x_1 x_0^2 - x_2 x_0^2 - x_1^2 x_0 + x_2^2 x_0 - x_1 x_2^2 + x_1^2 x_2
\]

Protože ten vzorec nevypadá úplně náhodně, mohlo by jít vytknout z tohoto výrazu
nějaký kratší výraz, třeba $x_0 - x_1$:

\[
  x_1 x_0^2 - x_2 x_0^2 - x_1^2 x_0 + x_2^2 x_0 - x_1 x_2^2 + x_1^2 x_2
   = (x_0 - x_1)(x_0 x_1 - x_0 x_2 - x_1 x_2 - x_2^2)
\]

Teď je vidět, že ho lze ještě rozložit:

\[
  \text{det} \begin{pmatrix}
    x_0^2, x_0, 1 \\ x_1^2, x_1, 1 \\ x_2^2, x_2, 1
  \end{pmatrix} = (x_0 - x_1)(x_0 - x_2)(x_1 - x_2)
\]

Ze zadání víme, že platí $x_0 < x_1 < x_2$, což jasně implikuje, že determinant
nemůže být nulový, protože determinant je nulový právě tehdy, když jsou některé
z x-ových souřadnic stejné. Tím je důkaz u konce.

\section{Úkol 3 -- Součin}

Součin matic značí provedení dvou lineárních zobrazení za sebou. Po
prvním zobrazení bude obsah $\text{det} \, \mathbf{A}$. A protože
víme, že determinant násobí původní obsah, po provedení zobrazení
$\mathbf{A} \mathbf{B}$ bude determinant $\text{det} \, \mathbf{A} 
\cdot \text{det} \, \mathbf{B}$. Proto musí platit vztah:

\[
  \text{det}(\mathbf{AB}) = \text{det} \, \mathbf{A} \cdot \text{det} \, \mathbf{B}
\]

\end{document}
